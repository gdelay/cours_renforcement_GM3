\documentclass[12pt]{article}
\usepackage[utf8]{inputenc}
\usepackage[T1]{fontenc}
\usepackage[frenchb]{babel}
\usepackage{amsfonts,amssymb,amsmath,amsthm}
\usepackage{xcolor}
\usepackage{multicol}
\usepackage{geometry}
\geometry{hmargin=3cm, vmargin=3cm}
%
\pagestyle{empty}
%
\newtheorem{exercice}{\bf Exercice}
\newtheorem{correction}{\bf Correction exercice}
\newenvironment{exo}{
%\vskip .1cm
\begin{exercice}\smallskip\normalfont}{\end{exercice}
%\vskip .1cm
}
\newenvironment{cor}{
%\vskip .1cm
\begin{correction}\smallskip\normalfont}{\end{correction}
%\vskip .1cm
}
 


\def\titre{\centerline{
\hfill \begin{tabular}{r}
{\bf Polytech Sorbonne - GM3} \hspace{3cm} 
{\bf Renforts en Math\'ematiques - Ann\'ee 2022-2023} \\ \\{}\end{tabular}     } 
}


\newif\ifcorrige\corrigetrue
%\newif\ifcorrige\corrigefalse 
\begin{document}
\titre
\begin{center}
  \underline{\LARGE Nombres complexes et trigonom\'etrie}
\end{center}
\vskip .5cm 
  
\begin{exo} Montrer les identit\'es suivantes. 
\begin{enumerate}
\item $\frac{\cos(\theta)}{1+\cot(\theta)}=\frac{\sin(\theta)}{1+\tan(\theta)}$
\item $\frac{\tan(\theta)}{\sqrt{1+\tan^2(\theta)}}=\sin(\theta),\ \theta \in ]0,\pi/2[$
\item $\cos^8(\theta)-\sin^8(\theta)=(\cos^2(\theta)-\sin^2(\theta))(1-2\sin^2(\theta)\cos^2(\theta))$
\end{enumerate}
\end{exo}

%%%%%%%%%%%%%%%%%
\ifcorrige
\color{magenta}
\begin{cor}
  $\qquad$
\begin{enumerate}
\item Multiplier haut et bas par $\sin$ puis factoriser par $\cos$.
\item Multiplier haut et bas par $\cos$.
\item Poser $a=\cos^2(\theta)$, $b=\sin^2(\theta)$, on a alors 
$$\cos^8(\theta)-\sin^8(\theta)=a^4-b^4=(a-b)(a^3+ba^2+b^2a+b^3)$$
puis utiliser plusieurs fois l'identit\'e $a+b=1$.
\end{enumerate}
\end{cor}
\color{black}
\fi


%%%%%%%%%%%%%%%%
\begin{exo}

\
\begin{enumerate} 
 \item Montrer que si $\cos(\theta)-\sin(\theta)=\sqrt{2}\sin(\theta)$ alors $\cos(\theta)+\sin(\theta)=\sqrt{2}\cos(\theta)$.
 \item Trouver l'image de la fonction $f(x)=a\cos^2(bx+c)+d$, en fonction des param\`etres $a,b,c,d \in \mathbb{R}$.
 \item Sachant que $\sin(\theta)=3/5$ et $0<\theta<\pi/2$, trouver $\cos(\theta)$ et $\tan(\theta)$.
 \item Sachant que $\tan \theta +\cot \theta=2$, trouver $\tan^n \theta +\cot^n\theta$ pour tout $n \in \mathbb{N}$.
 \item Sachant que $\tan \theta +\cot \theta=5$, trouver $\tan^4 \theta +\cot^4\theta$ .
 \item Exprimer $\cos(4\theta)$ comme un polyn\^ome en $\cos(\theta)$.
 \item Montrer que si $\alpha+\beta\neq \pi/2\  [\pi]$, alors
 $$\tan(\alpha+\beta)=\frac{\tan \alpha+\tan \beta}{1-\tan \alpha \tan \beta}.$$
 \item Montrer que si $A+B=\pi/4$, alors $(1+\tan A)(1+\tan B)=2$.
\end{enumerate}
\end{exo}

\ifcorrige
\color{magenta}
\begin{cor}
  $\qquad$
\begin{enumerate}
\item On a $\cos(\theta)=(1+\sqrt{2})\sin(\theta)$, donc 
$$\cos(\theta)+\sin(\theta)=\left(1+\frac{1}{1+\sqrt{2}}\right) \cos(\theta),$$
en r\'eduisant au m\^eme d\'enominateur et en multipliant par les quantit\'es conjugu\'ees on a cqfd.
\item Si $b=0$ ou $a=0$ c'est le singleton $\{a\cos(c)+d\}$, si $b\neq 0$ et $a\neq 0$, c'est l'intervale $[d,d+a]$ si $a>0$ et $[d+a,d]$ si $a<0$.
\item En utilisant $\cos^2\theta=1-\sin^2\theta=\frac{16}{25}$ on en d\'eduit ($0<\theta<\pi/2$) que $\cos \theta= \frac{4}{5}$ puis
$\tan\theta=\frac{3}{4}$.
\item Posons $x=\tan\theta$, on a
$$x+\frac{1}{x}=2,$$
donc $x=1$ puis $\tan^n \theta +\cot^n\theta=2$ pour tout $n$.
\item En utilisant la m\^eme id\'ee, on a $\tan\theta=\frac{5\pm\sqrt{21}}{2}$, $\cot \theta=\frac{5\mp\sqrt{21}}{2}$ et donc
$$\tan^4 \theta +\cot^4\theta= \left (\frac{5+\sqrt{21}}{2}\right)^4+  \left (\frac{5-\sqrt{21}}{2}\right)^4.$$
\item On a par la formule d'addition $\cos(2\theta)=\cos^2 \theta-\sin^2 \theta=2\cos^2\theta-1$,
donc $\cos(4\theta)=2\cos^2(2\theta)-1,$
puis en d\'eveloppant
$$\cos(4\theta)=8\cos^4 \theta-8\cos^2\theta+1. $$
\item D\'ecoule des formules d'addition pour $\sin$ et $\cos$.
\item Appliquer la question pr\'ec\'edente.
\end{enumerate}
\end{cor}
\color{black}
\fi

\begin{exo} Simplifiez les expressions suivantes.
\begin{enumerate}
\item $z=(2-3i)(3i)$.
 \item $z=(1+i)(2+i)$.
 \item $z=(2+3i)^2-i$.
 \item $z=\frac{1-i}{2i}$.
 \item $z=\frac{(1-i)}{(2+i)}-2i$.
 \item On pose $j=-\frac{1}{2}+i\frac{\sqrt{3}}{2}$, calculer $1+j+j^2$.
\end{enumerate}
 \end{exo}
 \ifcorrige
\color{magenta}
\begin{cor}
  $\qquad$
\begin{enumerate}
\item $z=9+6i$.
\item $z=1+3i$.
\item $z=-5-11i$.
\item $z=-\frac{1}{2}-\frac{i}{2}$.
\item $z=\frac{1}{5}-\frac{13i}{5}$.
\item On trouve $1+j+j^2=0$.
\end{enumerate}
\end{cor}
\color{black}
\fi

 
 
\begin{exo} Mettre les nombres complexes suivants sous forme exponentielle, simplifier.
 \begin{enumerate}
  \item $z=3+3i$. 
  \item $z=1-i$.
  \item  $z=-2+2i$.
  \item  $z=2\sqrt{3}-2i$.
  \item $z=\frac{\sqrt{2}}{(1-i)}$.
  \item $z=(-1+i)^3e^{3i\pi/4}$.
  \item $z=\left( \frac{1+i}{1-i} \right)^3$.
 \end{enumerate}
 \end{exo}
 
  \ifcorrige
\color{magenta}
\begin{cor}
  $\qquad$
\begin{enumerate}
\item $z=3\sqrt{2}e^{i\pi/4}$.
\item $z=\sqrt{2}e^{-i\pi/4}$.
\item $z=2\sqrt{2}e^{3i\pi/4}$.
\item $z=4e^{-i\pi/6}$.
\item $z=e^{i\pi/4}$.
\item $z=-2\sqrt{2}$.
\item $z=-i$.
\end{enumerate}
\end{cor}
\color{black}
\fi

 
 
\begin{exo}
 R\'esoudre, dans  $\mathbb{C}$, les \'equations suivantes.
 \begin{enumerate}
  \item $z^3=i$.
  \item $z^4=4e^{i\pi/3}$.
  \item $z^2-z+1=0$.
  \item $2z^2+z-3=0$.
  \item $z=2\overline{z}$.
  \item $z-\overline{z}=i$.
  \item $z\overline{z}-z-\overline{z}=3$.
 \end{enumerate}
 \end{exo}
 
   \ifcorrige
\color{magenta}
\begin{cor}
  $\qquad$
\begin{enumerate}
\item $z \in \{ e^{i\pi/6}, e^{i5\pi/6}, e^{-i\pi/2}  \}$.
\item $z \in \{\sqrt{2}e^{i\pi/12}, \sqrt{2}e^{i7\pi/12},   \sqrt{2}e^{i13\pi/12},  \sqrt{2}e^{-i5\pi/12}   \}$.
\item $z=\frac{1\pm i\sqrt{3}}{2}$.
\item $z=1$ ou $z=-3/2$.
\item En passant au module, on trouve $\vert z\vert=2\vert z\vert$ et donc la seule solution est $z=0$.
\item En posant $z=x+iy$ on trouve $y=1/2$, donc l'ensemble des solutions est la droite horizontale d'\'equation $y=1/2$.
\item En posant $z=x+iy$, on a
$$z\overline{z}-z-\overline{z}=3 \Leftrightarrow  x^2+y^2-2x=3 \Leftrightarrow (x-1)^2+y^2=4,$$
l'ensemble des solutions est le cercle centr\'e en $1$ et de rayon $2$.
\end{enumerate}
\end{cor}
\color{black}
\fi

 
 
 
 \begin{exo}
 R\'esoudre le syst\`eme lin\'eaire suivant dans $\mathbb{C}$.
 $$\left \{ z_1+iz_2=2 \atop -iz_1+2z_2=1       \right.$$
 \end{exo}
 
    \ifcorrige
\color{magenta}
\begin{cor}
  $\qquad$

En faisant $L_2 \leftarrow iL_1+L2$, on trouve $z_2=1+2i$ puis $z_1=4-i$.
\end{cor}
\color{black}
\fi

 
 
 
 \begin{exo}
  Utilisez la formule 
  $$e^{i(a+b)}=e^{ia} e^{ib}$$ 
  pour prouver les formules d'addition suivantes:
  \begin{enumerate}
   \item $\cos(a+b)=\cos(a)\cos(b)-\sin(a)\sin(b)$.
   \item $\sin(a+b)=\sin(a)\cos(b)+\cos(a)\sin(b)$.
  \end{enumerate}
 \end{exo}
 
    \ifcorrige
\color{magenta}
\begin{cor}
  $\qquad$
  
Utiliser la formule de De Moivre, d\'evelopper et identifier partie r\'eelle et imaginaire.
\end{cor}
\color{black}
\fi


\end{document}
