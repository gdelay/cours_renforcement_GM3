\documentclass[12pt]{article}
\usepackage[utf8]{inputenc}
\usepackage[T1]{fontenc}
\usepackage[frenchb]{babel}
\usepackage{amsfonts,amssymb,amsmath,amsthm}
\usepackage{xcolor}
\usepackage{multicol}
\usepackage{geometry}
\geometry{hmargin=3cm, vmargin=3cm}
%
\pagestyle{empty}
%
\newtheorem{exercice}{\bf Exercice}
\newtheorem{correction}{\bf Correction exercice}
\newenvironment{exo}{
%\vskip .1cm
\begin{exercice}\smallskip\normalfont}{\end{exercice}
%\vskip .1cm
}
\newenvironment{cor}{
%\vskip .1cm
\begin{correction}\smallskip\normalfont}{\end{correction}
%\vskip .1cm
}
 


\def\titre{\centerline{
\hfill \begin{tabular}{r}
{\bf Polytech Sorbonne - GM3} \hspace{3cm} 
{\bf Renforts en Math\'ematiques - Ann\'ee 2024-2025} \\ \\{}\end{tabular}     }
}


\newif\ifcorrige\corrigetrue
%\newif\ifcorrige\corrigefalse 
\begin{document}
\titre
\begin{center}
  \underline{\LARGE Calcul litt\'eral, racines et \'equations}
\end{center}
\vskip .5cm 
  
\begin{exo} Factoriser les expressions suivantes:
\begin{multicols}{3}
\begin{enumerate}
\item $6x^2 + 9x$
\item $12x^2 y + 8xy^2$
\item $15x - 20xy^2$
\item $xy^2 - x^2y$
\item $x^2y - xy^3$
\item $5x^2y - 25xy + 3yx^2$
\end{enumerate}
\end{multicols}
\end{exo}

%%%%%%%%%%%%%%%%%
\ifcorrige
\color{magenta}
\begin{cor}
  $\qquad$
\begin{multicols}{3}
\begin{enumerate}
\item $3x(2x+3)$
\item $4xy(3x+2y)$
\item $5x(3-4y^2)$
\item $xy(y-x)$
\item $xy(x-y^2)$
\item $xy(5x - 25 + 3y)$ 
\end{enumerate}
\end{multicols}
\end{cor}
\color{black}
\fi

%%%%%%%%%%%%%%%%

\begin{exo} D\'evelopper les expressions suivantes:
\begin{multicols}{2}
\begin{enumerate}
\item $7(x-2)+3(x+4)-6(x-2)$
\item $4x(x+3)-2x(3x-7)$
\item $3x^2-x(3-4x)+9$
\item $2x^2(3x+1)-4x^2(5x-3)$
\end{enumerate}
\end{multicols}
\end{exo}

%%%%%%%%%%%%%%%%%
\ifcorrige
\color{magenta}
\begin{cor}
  $\qquad$
\begin{multicols}{2}
\begin{enumerate}
\item $4x+10$
\item $4x^2+12x-6x^2+14x=-2x^2+26x$
\item $3x^2+4x^2-3x+9=7x^2-3x+9$
\item $6x^3+2x^2-20x^3+12x^2=-14x^3+14x^2$
\end{enumerate}
\end{multicols}
\end{cor}
\color{black}
\fi

%%%%%%%%%%%%%%%%

\begin{exo} Factoriser les expressions suivantes:
\begin{multicols}{4}
\begin{enumerate}
\item $x^2 + 8x + 12$
\item $x^2 - 8x + 12$
\item $x^2-3x-10$
\item $x^2-x-6$
\item $x^2+5x+6$
\item $2x^2+5x+2$
\item $2x^2-2x-24$
\item $2x^2+7x-15$
\end{enumerate}
\end{multicols}
\end{exo}


%%%%%%%%%%%%%%%%%
\ifcorrige
\color{magenta}
\begin{cor}
  $\qquad$
\begin{enumerate}
\item On calcule le discriminant
  $\Delta = 8^2 - 4\times 1 \times 12 = 64 - 48 = 16 > 0$.
  Ce polyn\^ome admet donc deux racines r\'eelles distinctes
  $\frac{-8 \pm \sqrt{\Delta}}{2}$ qui valent $-6$ et $-2$.
  La version factoris\'ee est donc
  \begin{equation*}
    x^2 + 8x + 12
    =
    (x+6)(x+2)
  \end{equation*}
  (vous pouvez d\'evelopper cette expression pour v\'erifier l'\'egalit\'e)
\item Par un changement de variable $y=-x$, on revient au polyn\^ome
  $y^2-8y+12 = (y+6)(y+2)$ de la question pr\'ec\'edente.
  La forme factoris\'ee est donc
  \begin{equation*}
    x^2 - 8x + 12
    =
    (x-6)(x-2)
  \end{equation*}
  (on peut aussi \`a nouveau calculer le discriminant)
\item On calcule le discriminant
  $\Delta = (-3)^2 -4 \times 1 \times (-10) = 49 > 0$.
  Ce polyn\^ome admet donc deux racines r\'eelles distinctes donn\'ees par
  $\frac{+3\pm \sqrt{\Delta}}{2}$ qui valent $-2$ et $5$.
  La forme factoris\'ee est donc
  \begin{equation*}
    x^2-3x-10 = (x+2)(x-5)
  \end{equation*}
\item Le discriminant vaut $\Delta = 1 - 4 \times (-6) = 25 > 0$.
  Ce polyn\^ome admet donc deux racines r\'eelles distinctes
  donn\'ees par $\frac{1 \pm 5}{2}$.
  La forme factoris\'ee est donc
  \begin{equation*}
    x^2-x-6 = (x+2)(x-3)
  \end{equation*}
\item Le discriminant vaut $\Delta = 25 - 24 = 1>0$.
  Ce polyn\^ome admet deux racines r\'eelles $-2$ et $-3$.
  Sa forme factoris\'ee est donc
  \begin{equation*}
    x^2+5x+6 = (x+2)(x+3)
  \end{equation*}
\item
  Le discriminant vaut $\Delta = 9 > 0$, ce polyn\^ome admet deux racines r\'eelles
  $\frac{-5 \pm 3}{2\times 2}$ (ne pas oublier $\times 2$ au d\'enominateur)
  qui valent $-2$ et $-\frac12$
  Le polyn\^ome vaut donc
  \begin{equation*}
    2x^2+5x+2 = 2(x+2)(x+\frac12)
  \end{equation*}
  (ne pas oublier de tout multiplier par le coefficient de plus haut degr\'e, ici $2$)
\item
  Ici, on peut simplifier un peu avec $2x^2-2x-24 = 2(x^2 -x-12)$
  et on factorise directement $x^2-x-12 = (x-4)(x+3)$.
  La forme factoris\'ee du polyn\^ome est donc
  \begin{equation*}
    2x^2-2x-24 = 2(x-4)(x+3)
  \end{equation*}
\item
  Le discriminant vaut $\Delta = 7^2 - 4 \times 2 \times (-15) = 169 = 13^2$
  Les racines valent $\frac{-7 \pm 13}{2 \times 2}$ donc $-5$ et $\frac32$.
  La forme factoris\'ee du polyn\^ome est donc
  \begin{equation*}
    2x^2+7x-15 = 2(x+5)(x-\frac32)
  \end{equation*}
\end{enumerate}
\end{cor}
\color{black}
\fi

%%%%%%%%%%%%%%%%

\begin{exo} Simplifier les expressions suivantes:
\begin{multicols}{2}
\begin{enumerate}
\item $2 \sqrt{18} - 4 \sqrt{72} - \sqrt{50} + 3\sqrt{98}$
\item $4\sqrt{8} -2\sqrt{75} + \sqrt{200}-3\sqrt{48}+5\sqrt{45}$
\item $\frac1{\sqrt{5}-2}$
\item $\frac{\sqrt{3}+\sqrt{2}}{\sqrt{3}-\sqrt{2}}$
\item $\frac1{4\sqrt{11} -5\sqrt{7}}$
\item $\frac{5x^2 - 125}{x^2 + 5x} \div \frac{10x^2 + 40x - 50}{3x^2}$
\item $\frac{3}{x^2+1} - \frac{1}{x-1} + \frac{2}{(x-1)^2}$
\item $\frac{2}{x-2} - \frac{3x+1}{x^2-7x+10} - \frac{1}{x-5}$
\end{enumerate}
\end{multicols}
\end{exo}

%%%%%%%%%%%%%%%%%
\ifcorrige
\color{magenta}
\begin{cor}
  $\qquad$
\begin{enumerate}
\item Ici, on veut r\'eduire les nombres sous les racines.
  On utilise $\sqrt{18} = \sqrt{9} \times \sqrt{2} = 3\sqrt{2}$,
  $\sqrt{72} = \sqrt{4} \times \sqrt{18} = \sqrt{4} \times \sqrt{9} \times \sqrt{2} = 6\sqrt{2}$,
  $\sqrt{50} = \sqrt{25} \sqrt{2} = 5 \sqrt{2}$,
  $\sqrt{98} = \sqrt{2} \times \sqrt{49} = 7 \sqrt{2}$.
  On a
  \begin{align*}
    2 \sqrt{18} - 4 \sqrt{72} - \sqrt{50} + 3\sqrt{98}
    &= 6 \sqrt{2} - 24 \sqrt{2} - 5 \sqrt{2} +  21\sqrt{2}
    \\
    &= -2 \sqrt{2}
  \end{align*}
\item De m\^eme
  $\sqrt{8} = 2 \sqrt{2}$, $\sqrt{75} = 5 \sqrt{5}$, $\sqrt{200} = 10 \sqrt{2}$,
  $\sqrt{48} = 4 \sqrt{3}$ et $\sqrt{45} = 3 \sqrt{5}$.
  Ainsi, on a
  \begin{align*}
    4\sqrt{8} -2\sqrt{75} + \sqrt{200}-3\sqrt{48}+5\sqrt{45}
    &= 8 \sqrt{2} - 10 \sqrt{5} + 10 \sqrt{2} - 12 \sqrt{3} +15 \sqrt{5}
    \\
    &= 18 \sqrt{2} - 12 \sqrt{3} + 5 \sqrt{5}
  \end{align*}
\item Ici, on veut retirer les racines carr\'ees des d\'enominateurs.
  On va multiplier le num\'erateur et le d\'enominateur par le "conjugu\'e" du d\'enominateur
  (cf nombres complexes) : le conjugu\'e de $\sqrt{5}-2$ est $\sqrt{5}+2$
  \begin{align*}
    \frac1{\sqrt{5}-2}
    = \frac{\sqrt{5}+2}{(\sqrt{5}-2)(\sqrt{5}+2)}= \frac{\sqrt{5}+2}{\sqrt{5}^2-2^2}
    = \sqrt{5}+2
  \end{align*}
\item Comme pr\'ec\'edemment
  \begin{align*}
    \frac{\sqrt{3}+\sqrt{2}}{\sqrt{3}-\sqrt{2}}
    &= \frac{(\sqrt{3}+\sqrt{2})(\sqrt{3}+\sqrt{2})}{(\sqrt{3}-\sqrt{2})(\sqrt{3}+\sqrt{2})}
      = \frac{(\sqrt{3}+\sqrt{2})^2}{3-2}
      = (\sqrt{3}+\sqrt{2})^2
  \end{align*}

\item
  Comme pr\'ec\'edemment
  \begin{align*}
    \frac1{4\sqrt{11} -5\sqrt{7}}
    &= \frac{4\sqrt{11}+5\sqrt{7}}{(4\sqrt{11} -5\sqrt{7})(4\sqrt{11} +5\sqrt{7})}
    \\
    &= \frac{4\sqrt{11}+5\sqrt{7}}{16 \times 11 - 25 \times 7}
    \\
    &= \frac{4\sqrt{11}+5\sqrt{7}}{176 - 175}
    \\
    &= 4\sqrt{11}+5\sqrt{7}
  \end{align*}

\item Ici, on simplifie en \'ecrivant tout avec une seule fraction.
  Puis, on simplifie cette fraction en factorisant et en simplifiant les facteurs communs
  entre num\'erateur et d\'enominateur.
  \begin{align*}
    \frac{5x^2 - 125}{x^2 + 5x} \div \frac{10x^2 + 40x - 50}{3x^2}
    &= \frac{(5x^2 - 125)3x^2}{(x^2 + 5x)(10x^2 + 40x - 50)}
    \\
    &= \frac{15(x^2 - 25)x^2}{10x(x + 5)(x^2 + 4x - 5)}
    \\
    &= \frac{3(x-5)(x+5)x^2}{2x(x + 5)(x^2 + 4x - 5)}
    \\
    &= \frac{3(x-5)x}{2(x^2 + 4x - 5)}
  \end{align*}
  On peut encore simplifier un peu en factorisant le d\'enominateur.
  Le discriminant de $x^2 + 4x - 5$ est $\Delta = 36 > 0$
  On a donc $x^2 + 4x - 5 = (x+5)(x-1)$.
  Finalement
  \begin{align*}
    \frac{5x^2 - 125}{x^2 + 5x} \div \frac{10x^2 + 40x - 50}{3x^2}
    = \frac{3(x-5)x}{2(x+5)(x-1)}
  \end{align*}
  
\item Dans ce cas, on simplifie en regroupant les diff\'erentes fractions en une seule fraction.
  Il faut donc tout mettre sous le m\^eme d\'enominateur.
  On peut ensuite d\'evelopper le num\'erateur pour obtenir un seul monome de chaque degr\'e.
  \begin{align*}
    \frac{3}{x^2+1} - \frac{1}{x-1} + \frac{2}{(x-1)^2}
    &= \frac{3(x-1)^2}{(x^2+1)(x-1)^2} - \frac{(x-1)(x^2+1)}{(x^2+1)(x-1)^2}
      + \frac{2 (x^2+1)}{(x^2+1)(x-1)^2}
    \\
    &= \frac{3(x-1)^2 - (x-1)(x^2+1) + 2 (x^2 + 1 ) }{(x^2+1)(x-1)^2}
    \\
    &= \frac{3(x^2-2x+1) - (x^3-x^2+x-1) + 2 x^2 + 2}{(x^2+1)(x-1)^2}
    \\
    &= \frac{-x^3 + 6 x^2 - 7 x + 6}{(x^2+1)(x-1)^2}
  \end{align*}
  A ce stade, on se demande si on ne pourrait pas factoriser le num\'erateur.
  Pour un polyn\^ome de degr\'e 3 ou plus, on ne sait pas faire autrement qu'en cherchant
  une racine \'evidente (essayer $-2$, $-1$, $0$, $1$ et $2$).
  Aucun de ces nombres n'est racine. On ne peut donc pas simplifier davantage.

\item On applique la m\^eme m\'ethode qu'\`a la question pr\'ec\'edente.
  De plus, pour simplifier, on va factoriser le polyn\^ome qui ne l'est pas au d\'enominateur.
  On a
  \begin{align*}
    x^2-7x+10 = (x-5)(x-2)
  \end{align*}
  Et donc
  \begin{align*}
    \frac{2}{x-2} - \frac{3x+1}{x^2-7x+10} - \frac{1}{x-5}
    &= \frac{2}{x-2} - \frac{3x+1}{(x-5)(x-2)} - \frac{1}{x-5}
    \\
    &= \frac{2(x-5)}{(x-2)(x-5)} - \frac{3x+1}{(x-5)(x-2)} - \frac{x-2}{(x-2)(x-5)}
    \\
    &= \frac{2(x-5) - (3x+1) - (x-2)}{(x-2)(x-5)}
    \\
    &= \frac{-2x - 9}{(x-2)(x-5)}
  \end{align*}
\end{enumerate}
\end{cor}
\color{black}
\fi

%%%%%%%%%%%%%%%%


\begin{exo} Trouver les solutions r\'eelles (\'eventuelles) des \'equations suivantes:
\begin{multicols}{2}
\begin{enumerate}
\item $x^2 - 4x - 8 = 0$
\item $x^2 + 2x - 5 = 0$
\item $x^2 + 2x + 1 = 0$
\item $3x^2 + 3x + 1 = 0$
\end{enumerate}
\end{multicols}
\end{exo}

%%%%%%%%%%%%%%%%%
\ifcorrige
\color{magenta}
\begin{cor}
  $\qquad$
\begin{enumerate}
\item Les solutions de cette \'equation sont les racines r\'eelles du polyn\^ome.
  Le discriminant vaut $\Delta = 48 > 0$.
  Cette \'equation admet donc deux solutions r\'eelles
  $x_1 = \frac{4 - \sqrt{48}}{2} = \frac{4 - 4\sqrt{3}}{2} = 2 - 2\sqrt{3}$
  et
  $x_2 = \frac{4 + \sqrt{48}}{2} = \frac{4 + 4\sqrt{3}}{2} = 2 + 2\sqrt{3}$.
\item Le discriminant vaut $\Delta = 24 > 0$.
  L'\'equation admet donc deux solutions r\'eelles:
  $x_1 = \frac{-2 - \sqrt{24}}{2} = -1 - \sqrt{6}$
  et $x_2 = \frac{-2 + \sqrt{24}}{2} = -1 + \sqrt{6}$.
\item Dans ce cas, le discriminant vaut $\Delta = 0$.
  L'\'equation admet une unique solution r\'eelle : $x = -1$.
\item Dans ce cas, le discriminant vaut $\Delta = -3 < 0$
  et l'\'equation n'admet pas de solution r\'eelle.
\end{enumerate}
\end{cor}
\color{black}
\fi

%%%%%%%%%%%%%%%%

\begin{exo} Trouver les solutions des syst\`emes lin\'eaires suivants:
\begin{multicols}{3}
\begin{enumerate}
\item
  $
    \left\{
      \begin{array}{l}
        x + y = 2
        \\
        x - y = 5
      \end{array}
    \right.
  $
\item
  $
    \left\{
      \begin{array}{l}
        3x + y = 5
        \\
        x + 3y = 7
      \end{array}
    \right.
    $
    \item
  $
    \left\{
      \begin{array}{l}
        5x + 2y = 11
        \\
        -x -4y = 5
      \end{array}
    \right.
  $
\end{enumerate}
\end{multicols}
\end{exo}

%%%%%%%%%%%%%%%%%
\ifcorrige
\color{magenta}
\begin{cor}
  $\qquad$
\begin{enumerate}
\item On note (a) et (b) les deux lignes du syst\`eme.
  On injecte (a) dans (b) (combinaison) dans le but d'\'eliminer les $y$,
  on trouve la valeur de $x$ puis
  on utilise cette valeur pour trouver $y$.
  \begin{align*}
    \left\{
      \begin{array}{l}
        x + y = 2
        \\
        x - y = 5
      \end{array}
    \right.
    \iff
    \left\{
      \begin{array}{l}
        x + y = 2
        \\
        2x = 7
      \end{array}
    \right.
    \iff
    \left\{
      \begin{array}{l}
        x + y = 2
        \\
        x = \frac72
      \end{array}
    \right.
    \iff
    \left\{
      \begin{array}{l}
        y = 2 - x = 2 - \frac72 = - \frac32
        \\
        x = \frac72
      \end{array}
    \right.
  \end{align*}
\item
  On applique la m\^eme m\'ethode (on injecte $-3 \times$ (a) dans (b))
  \begin{align*}
    \left\{
      \begin{array}{l}
        3x + y = 5
        \\
        x + 3y = 7
      \end{array}
    \right.
    \iff
    \left\{
      \begin{array}{l}
        3x + y = 5
        \\
        -8x  = -8
      \end{array}
    \right.
    \iff
    \left\{
      \begin{array}{l}
        3x + y = 5
        \\
        x  = 1
      \end{array}
    \right.
    \iff
    \left\{
      \begin{array}{l}
        y = 5 -3x = 2
        \\
        x  = 1
      \end{array}
    \right.
  \end{align*}
  \item
  On applique la m\^eme m\'ethode (on injecte $2 \times$ (a) dans (b))
  \begin{align*}
    \left\{
      \begin{array}{l}
        5x + 2y = 11
        \\
        -x - 4y = 5
      \end{array}
    \right.
    &\iff
    \left\{
      \begin{array}{l}
        5x + 2y = 11
        \\
        9x  = 27
      \end{array}
    \right.
    \iff
    \left\{
      \begin{array}{l}
        5x + 2y = 11
        \\
        x  = 3
      \end{array}
    \right.
    \\
    &\iff
    \left\{
      \begin{array}{l}
        2y = 11 - 5x = -4
        \\
        x  = 3
      \end{array}
    \right.
    \iff
    \left\{
      \begin{array}{l}
        y = -2
        \\
        x  = 3
      \end{array}
    \right.
  \end{align*}
\end{enumerate}
\end{cor}
\color{black}
\fi

%%%%%%%%%%%%%%%%

\end{document}
