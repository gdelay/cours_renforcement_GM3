\documentclass[12pt]{article}
\usepackage[utf8]{inputenc}
\usepackage[T1]{fontenc}
\usepackage[frenchb]{babel}
\usepackage{amsfonts,amssymb,amsmath,amsthm}
\usepackage{xcolor}
\usepackage{multicol}
\usepackage{geometry}
\usepackage{tikz}
\usepackage{pgfplots} % ln in tikz
\geometry{hmargin=3cm, vmargin=3cm}
%
\pagestyle{empty}
%
\newtheorem{exercice}{\bf Exercice}
\newtheorem{correction}{\bf Correction exercice}
\newenvironment{exo}{
%\vskip .1cm
\begin{exercice}\smallskip\normalfont}{\end{exercice}
%\vskip .1cm
}
\newenvironment{cor}{
%\vskip .1cm
\begin{correction}\smallskip\normalfont}{\end{correction}
%\vskip .1cm
}
 

\newcommand*{\R}{\mathbb{R}}

\def\titre{\centerline{
\hfill \begin{tabular}{r}
{\bf Polytech Sorbonne - GM3} \hspace{3cm} 
{\bf Renforts en Math\'ematiques - Ann\'ee 2022-2023} \\ \\{}\end{tabular}     } 
}


%\newif\ifcorrige\corrigetrue
\newif\ifcorrige\corrigefalse 
\begin{document}
\titre
\begin{center}
  \underline{\LARGE Fonctions r\'eelles d'une variable r\'eelle}
\end{center}
\vskip .5cm 
  
\begin{exo} Donner le domaine de d\'efinition des fonctions suivantes:
\begin{multicols}{5}
\begin{enumerate}
\item $\frac1{1+x}$
\item $\frac{3}{1+x^2}$
\item $e^{\frac{1}{1-x}}$
\item $\ln(1-x)$
\item $\sqrt{-4x}$
\end{enumerate}
\end{multicols}
\end{exo}

%%%%%%%%%%%%%%%%%
\ifcorrige
\color{magenta}
\begin{cor}
  $\qquad$
\begin{multicols}{3}
\begin{enumerate}
\item $\R \setminus \{ -1 \}$
\item $\R$
\item $\R \setminus \{ 1 \}$
\item $]-\infty , 1[$
\item $]- \infty, 0]$
\end{enumerate}
\end{multicols}
\end{cor}
\color{black}
\fi

%%%%%%%%%%%%%%%%

\begin{exo} Calculer l'ensemble image des fonctions suivantes:
\begin{multicols}{2}
\begin{enumerate}
\item $f(x) = 3x-2$ avec $D_f = \{ 1,2,3,4 \}$
\item $g(x) = x^2$ avec $D_g = [-5,5]$
\end{enumerate}
\end{multicols}
\end{exo}

%%%%%%%%%%%%%%%%%
\ifcorrige
\color{magenta}
\begin{cor}
  $\qquad$
\begin{multicols}{2}
\begin{enumerate}
\item $\text{Im}(f) = \{1,4,7,10\}$
\item $\text{Im}(g) = [0,25]$
\end{enumerate}
\end{multicols}
\end{cor}
\color{black}
\fi

%%%%%%%%%%%%%%%%

\begin{exo} Soit la fonction $f : \R \to \R$, $f(x) = 2x^2+3$.
\begin{multicols}{2}
\begin{enumerate}
\item Calculer l'ensemble image de $f$.
\item Trouver $z\in\R$ tel que $f(z) = 35$.
\end{enumerate}
\end{multicols}
\end{exo}

%%%%%%%%%%%%%%%%%
\ifcorrige
\color{magenta}
\begin{cor}
  $\qquad$
\begin{enumerate}
\item Lorsque $x$ parcourt $\R$, $2x^2$ parcourt $[0,+\infty[$ et donc $2x^2+3$ parcourt $[3,+\infty[$.
  On a $\text{Im}(f) = [3,+\infty[$.
\item Commen\c{c}ons par remarquer que $35 \in \text{Im}(f)$.
  Il existe donc au moins un $z \in \R$ satisfaisant $f(z) = 35$.
  On r\'esout l'\'equation suivante :
  \begin{align*}
    f(z) = 35
    \iff
    2z^2+3 = 35
    \iff
    2z^2 = 32
    \iff
    z^2 = 16
    \iff
    z = 4 \text{  ou  } z = -4
  \end{align*}
  Il existe donc deux $z \in \R$ tels que $f(z) = 35$ :
  $z=-4$ et $z=4$.
\end{enumerate}
\end{cor}
\color{black}
\fi

%%%%%%%%%%%%%%%%

\begin{exo} Soient les fonctions $f(x) = (x+1)(x-2)$
  et $g(x) = 2x$.
\begin{multicols}{2}
\begin{enumerate}
\item Calculer $f \circ g$.
\item Calculer $g \circ f$.
\end{enumerate}
\end{multicols}
\end{exo}

%%%%%%%%%%%%%%%%%
\ifcorrige
\color{magenta}
\begin{cor}
  $\qquad$
\begin{enumerate}
\item On a
    $(f \circ g) (x) = f(g(x)) = (g(x)+1)(g(x)-2) = (2x+1)(2x-2)$
\item On a
    $(g \circ f) (x) = g(f(x)) = 2 f(x) = 2(x+1)(x-2)$
\end{enumerate}
\end{cor}
\color{black}
\fi

%%%%%%%%%%%%%%%%


\begin{exo} Soient les fonctions $f(x) = x^2-1$, $g(x) = 3x+2$, et $h(x) = \frac1x$.
  R\'esoudre les \'equations suivantes sur $\R$:
\begin{multicols}{3}
\begin{enumerate}
\item $(f \circ g)(x) = 15$
\item $(g \circ g)(x) = h(x)$
\item $(g \circ h)(x) = -4$
\end{enumerate}
\end{multicols}
\end{exo}

%%%%%%%%%%%%%%%%%
\ifcorrige
\color{magenta}
\begin{cor}
  $\qquad$
\begin{enumerate}
\item On r\'esout ce probl\`eme sur $\R$
  \begin{align*}
    (f \circ g)(x) = 15
    &\iff
    (g(x))^2-1 = 15
    \iff
    (3x+2)^2-1 = 15
    \\
    &\iff
    (3x+2)^2 = 16
    \iff
    3x+2 = -4 \text{  ou  } 3x+2 = 4
    \\
    &\iff 3x = -6 \text{  ou  } 3x = 2
    \iff x = -2 \text{  ou  } x = \frac23
  \end{align*}
  On a trouv\'e deux solutions \`a cette \'equation : $\{-2,\frac23\}$.
  V\'erifions ce r\'esultat:
  $(f \circ g)(-2) = f(g(-2)) = f(-4) = 15$
  et $(f \circ g)(\frac23) = f(g(\frac23)) = f(4) = 15$
\item On r\'esout ce probl\`eme sur $\R \setminus \{ 0 \}$
  (car $h$ non d\'efinie en $0$).
  \begin{align*}
    (g \circ g)(x) = h(x)
    &\iff
    3(3x+2)+2 = \frac1x
    \iff
    9x+8 = \frac1x
    \\
    &\iff
    (9x+8)x = 1
    \iff
    9x^2+8x -1 = 0
  \end{align*}
  Cherchons les racines \'eventuelles de ce polyn\^ome.
  On calcule son discriminant:
  $\Delta = 64 - 4 \times 9 \times (-1) = 100 > 0$.
  Ce polyn\^ome admet donc deux racines r\'eelles:
  $x_1 = \frac{-8 -10}{2 \times 9} = -1$ et $x_2 = \frac{-8+10}{2 \times 9} = \frac19$.
  Ces racines sont non nulles (rappelez-vous qu'on cherche des solutions sur $\R \setminus \{ 0 \}$),
  elles sont donc solutions de l'\'equation.

\item On cherche des solutions sur $\R^*$ (car $h$ est d\'efinie sur $\R^*$)
  \begin{align*}
    (g \circ h)(x) = -4
    \iff
    \frac3x + 2 = -4
    \iff
    \frac3x = -6
    \iff
    \frac1x = -2
    \iff
    x = - \frac12
  \end{align*}
  et $- \frac12 \in \R^*$, donc cette \'equation admet pour unique solution $- \frac12$.
\end{enumerate}
\end{cor}
\color{black}
\fi

%%%%%%%%%%%%%%%%


\begin{exo} Soit la fonction $f(x) = \sqrt{2x-1}$.
\begin{multicols}{2}
\begin{enumerate}
\item Donner $D_f$.
\item Calculer $f^{-1}(x)$.
\end{enumerate}
\end{multicols}
\end{exo}

%%%%%%%%%%%%%%%%%
\ifcorrige
\color{magenta}
\begin{cor}
  $\qquad$
\begin{enumerate}
\item La fonction $y \mapsto \sqrt{y}$ est d\'efinie pour $y \geq 0$.
  Dans notre cas, $y = 2x-1$ et $2x-1 \geq 0 \iff x \geq \frac12$.
  Donc $D_f = [ \frac12 , + \infty [$.
\item Calculons maintenant (si elle existe) l'inverse de $f$.
  Pour $y \in \R$, cherchons $x \geq \frac12$ tel que
  \begin{align*}
    y = f(x)
    \iff
    y = \sqrt{2x-1}
    \iff
    y^2 = 2x-1
    \iff
    x = \frac{y^2+1}{2}
  \end{align*}
  De plus, on a bien $\frac{y^2+1}{2} \geq 0$.
  Ainsi, pour chaque $y \in \R$, il existe un unique $x = \frac{y^2+1}{2} \in D_f$ tel que $f(x) = y$.
  La fonction $f$ est donc inversible et son inverse a pour expression $f^{-1}(x) = \frac{x^2+1}{2}$.
\end{enumerate}
\end{cor}
\color{black}
\fi

%%%%%%%%%%%%%%%%

\begin{exo} Soient les fonctions $f(x) = 3x+2$, $g(x) = \frac1x$, $x \neq 0$.
\begin{enumerate}
\item 
\begin{multicols}{3}
\begin{enumerate}
\item Donner $f^{-1}(x)$.
\item Donner $g^{-1}(x)$.
\item Donner $(g \circ f)^{-1}(x)$.
\end{enumerate}
\end{multicols}
\item V\'erifier que $(g \circ f)^{-1}(x) = (f^{-1} \circ g^{-1})(x) = \frac13 (\frac1x - 2)$.
\end{enumerate}
\end{exo}

%%%%%%%%%%%%%%%%%
\ifcorrige
\color{magenta}
\begin{cor}
  $\qquad$
\begin{enumerate}
\item
\begin{enumerate}
\item Soit $y \in \R$, cherchons $x \in \R$ tel que
  \begin{align*}
    f(x) = y
    \iff
    3x+2 = y
    \iff
    3x = y-2
    \iff
    x = \frac{y-2}{3}
  \end{align*}
  Pour chaque $y \in R$, il existe un unique $x = \frac{y-2}{3} \in \R$ tel que $f(x) = y$.
  La fonction $f$ est donc inversible et on a $f^{-1}(x) = \frac{x-2}{3}$.
\item Soit $y \in \R^*$, cherchons $x \in \R^*$ tel que
  \begin{align*}
    y = g(x)
    \iff
    y = \frac1x
    \iff
    x = \frac1y
  \end{align*}
  Pour chaque $y \in \R$, il existe un unique $x = \frac{1}{y} \in \R^*$ tel que $g(x) = y$.
  La fonciton $g$ est donc inversible et on a $g^{-1}(x) = \frac1x$.

\item On calcule $(g \circ f) (x) = \frac{1}{3x+2}$.
  Cette fonction est d\'efinie sur $\R \setminus \{ -\frac23 \}$ et est \`a valeurs dans $\R \setminus \{ 0 \}$.
  Soit $y \in \R \setminus \{ 0 \}$, on cherche $x \in \R \setminus \{ -\frac23 \}$
  tel que
  \begin{align*}
    (g \circ f) (x) = y
    \iff
    \frac{1}{3x+2} = y
    \iff
    3x+2 = \frac1y
    \iff
    3x = \frac1y - 2
    \iff
    x = \frac1{3y} - \frac23
  \end{align*}
  Pour chaque $y \in \R \setminus \{ 0 \}$, il existe un unique $x = \frac1{3y} - \frac23$
  tel que $y = (g \circ f) (x)$.
  De plus, pour $y \neq 0$, on a bien $x = \frac1{3y} - \frac23 \neq - \frac23$.
  La fonction $g \circ f$ est donc inversible et son inverse est donn\'ee par
  $(g \circ f)^{-1}(x) = \frac{1}{3x} - \frac23$.
\end{enumerate}
\item Il ne reste qu'\`a v\'erifier
  $(f^{-1} \circ g^{-1})(x) = \frac{1}{3x} - \frac23$.
  Le r\'esultat vient rapidement.
\end{enumerate}
\end{cor}
\color{black}
\fi

%%%%%%%%%%%%%%%%

\begin{exo} Simplifier les expressions suivantes:
\begin{multicols}{2}
\begin{enumerate}
\item $\log(18) - \log(24) - \log(2)$
\item $\ln(2) + \ln(3x) - \ln (2x)$
\item $\ln(3x^2) + \ln(2x) - \ln (6x^3)$
\item $\log (5x^2) - \log(10x^2) + \log(4x)$
\end{enumerate}
\end{multicols}
\end{exo}

%%%%%%%%%%%%%%%%%
\ifcorrige
\color{magenta}
\begin{cor}
Dans tout cet exercice, on utilise $\ln(a \times b) = \ln a + \ln b$ et $\ln(a^n) = n \ln a$
(et les m\^emes relations sont valables avec $\log$ \`a la place de $\ln$).
\begin{enumerate}
\item
  \begin{align*}
    \log(18) - \log(24) - \log(2)
    &=
    \log(2\times 3^2) - \log(2^3 \times 3) - \log(2)
    \\
    &= \log 2 + \log(3^2) - \log (2^3) - \log 3 - \log 2
    \\
    &= \log 2 + 2 \log 3 - 3 \log 2 - \log 3 - \log 2
    \\
    &= -3 \log 2 + \log 3
  \end{align*}
\item
  \begin{align*}
    \ln(2) + \ln(3x) - \ln (2x)
    = \ln(2) + \ln(3) + \ln(x) - \ln (2) - \ln(x)
    = \ln(3)
  \end{align*}

\item
  \begin{align*}
    \ln(3x^2) + \ln(2x) - \ln (6x^3)
    &= \ln(3) + 2\ln(x) + \ln(2) + \ln(x) - \ln(6) - 3 \ln(x)
    \\
    &= \ln(3) + \ln(2) - \ln(2 \times 3)
    \\
    &= 0
  \end{align*}

\item
  \begin{align*}
    \log (5x^2) - \log(10x^2) + \log(4x)
    &= \log(5) + 2 \log(x) - \log(10) - 2 \log(x) + \log(4) + \log (x)
    \\
    &= \log(x) + \log(5) - \log(2 \times 5) + \log(2^2)
    \\
    &= \log(x) + \log(2)
  \end{align*}
\end{enumerate}
\end{cor}
\color{black}
\fi

%%%%%%%%%%%%%%%%

\begin{exo} R\'esoudre les \'equations suivantes:
\begin{multicols}{3}
\begin{enumerate}
\item $2 \ln (x) + 1 = 5$
\item $\ln(2x+1) = 5$
\item $\frac14 \ln (4-3x) = 2$
\item $\ln( e^{2x-1} ) = 36$
\item $e^{2x+3} = 4$
\item $e^{-2x} + 10 = 24$
\item $e^{4x+5} = -4$
\item $e^{2x} - 5 e^{x} + 6 = 0$
\item $e^x + e^{-x} = 2$
\end{enumerate}
\end{multicols}
\end{exo}

%%%%%%%%%%%%%%%%%
\ifcorrige
\color{magenta}
\begin{cor}
$\qquad$
\begin{enumerate}
\item Le domaine de d\'efinition de $2 \ln (x) + 1$ est $\R^*_+$.
  On cherche donc $x$ dans $\R_+^*$ tel que
  \begin{align*}
    2 \ln (x) + 1 = 5
    \iff
    2 \ln (x) = 4
    \iff
    \ln(x) = 2
    \iff
    x = e^{2}
  \end{align*}
  (l'\'equivalence lors de l'application de $\exp$ vient du fait que la fonction exponentielle
  est une bijection de $\R$ vers $\R_+^*$)

  Il existe une unique solution $e^2 \in \R_+^*$ \`a cette \'equation.

\item Le domaine de d\'efinition de $\ln(2x+1)$ est $]-\frac12 , + \infty[$.
  On cherche donc $x > -\frac12$ tel que
  \begin{align*}
    \ln(2x+1) = 5
    \iff
    2x+1 = e^5
    \iff
    2x = e^5 - 1
    \iff
    x = \frac{e^5 - 1}{2}
  \end{align*}
  (l'\'equivalence lors de l'application de $\exp$ vient du fait que la fonction exponentielle
  est une bijection de $\R$ vers $\R_+^*$)

\item La fonction $\frac14 \ln (4-3x)$ est d\'efinie sur $]-\infty , \frac43 [$.
  On cherche $x < \frac43$ tel que
  \begin{align*}
    \frac14 \ln (4-3x) = 2
    \iff
    \ln (4-3x) = 8
    \iff
    4 - 3x = e^8
    \iff
    -3x = e^8 - 4
    \iff
    x = \frac43 - \frac{e^8}{3}
  \end{align*}

\item $e^{2x-1}$ est \`a valeur dans $\R_+^*$ o\`u $\ln$ est d\'efinie.
  La fonction $\ln( e^{2x-1} )$ est d\'efinie sur $\R$ et on a $\ln( e^{2x-1} ) = 2x-1$.
  On cherche $x \in \R$ tel que
  \begin{align*}
    \ln( e^{2x-1} ) = 36
    \iff
    2x-1 = 36
    \iff
    x = \frac{37}2
  \end{align*}

\item
  \begin{align*}
    e^{2x+3} = 4
    \iff
    2x+3 = \ln(4)
    \iff
    x = \frac{\ln(4) - 3}{2}
  \end{align*}
  (l'\'equivalence lors de l'application de $\ln$ vient du fait que cette fonction
  est une bijection de $\R_+^*$ vers $\R$)

\item
  \begin{align*}
    e^{-2x} + 10 = 24
    \iff
    e^{-2x} = 14
    \iff
    -2x = \ln(14)
    \iff
    x = - \frac12 \ln(14)
  \end{align*}
  (l'\'equivalence lors de l'application de $\ln$ vient du fait que cette fonction
  est une bijection de $\R_+^*$ vers $\R$)

\item La fonction exponentielle est \`a valeurs dans $\R_+^*$ et $-4 \notin \R_+^*$.
  Il n'existe donc aucun $x \in \R$ tel que $e^{4x+5} = -4$ : cette \'equation n'a pas de solution.

\item On effectue le changement de variables $t = e^x$.
  \begin{align*}
    e^{2x} - 5 e^{x} + 6 = 0
    \iff
    t^2 - 5 t + 6 = 0
  \end{align*}
  On calcule le discriminant de ce polyn\^ome:
  $\Delta = 25 - 24 = 1 > 0$.
  Ce polyn\^ome admet donc deux racines r\'eelles
  $t_1 = \frac{5-1}{2} = 2$ et $t_2 = \frac{5+1}{2} = 3$.
  Donc 
  \begin{align*}
    e^{2x} - 5 e^{x} + 6 = 0
    \iff
    e^x = 2 \text{  ou  } e^x = 3
    \iff
    x = \ln(2) \text{  ou  } x = \ln(3)
  \end{align*}
  Cette \'equation admet deux solutions : $\ln(2)$ et $\ln(3)$.
\item On effectue le changement de variables $t = e^x$.
  \begin{align*}
    e^x + e^{-x} = 2
    \iff
    e^{2x} - 2 e^x + 1 = 0
    \iff
    t^2 - 2 t + 1 = 0
  \end{align*}
  On calcule le discriminant de ce polyn\^ome :
  $\Delta = 0$. Il admet donc une unique racine (double) : $t_0 = -1$.
  Donc
  \begin{align*}
    e^x + e^{-x} = 2
    \iff
    e^x = -1
  \end{align*}
  On a $e^x > 0$ et donc on ne peut pas avoir $e^x = -1$.
  Cette \'equation n'admet donc aucune solution.
\end{enumerate}
\end{cor}
\color{black}
\fi

%%%%%%%%%%%%%%%%

\begin{exo} Tracer les graphes des fonctions suivantes:
\begin{multicols}{5}
\begin{enumerate}
\item $\ln(x)+1$
\item $\ln(x-2)$
\item $\ln(-x)$
\item $\ln(x+2)$
\item $\ln(1-x)$
\end{enumerate}
\end{multicols}
\end{exo}

%%%%%%%%%%%%%%%%%
\ifcorrige
\color{magenta}
\begin{cor}
  \begin{center}
  \begin{tikzpicture}[scale=2]
    %\draw [help lines] (-1,-2) grid [step=1] (3,4);
    \draw[->] (-1,0) -- (3,0) node[right] {$x$};
    \draw[->] (0,-2) -- (0,3) node[above] {$y$};
    \draw[domain=0.05:3,variable=\x, green] plot ({\x},{ln(\x)+1}) node[above] {$y=\ln(x)+1$};
    \draw[domain=2.05:5,variable=\x, blue] plot ({\x},{ln(\x-2)}) node[above] {$y=\ln(x-2)$};
    \draw[domain=-3:-0.05,variable=\x, red] plot ({\x},{ln(-\x)}) node[left] {$y=\ln(-x)$};
    \draw[domain=-1.95:1,variable=\x, violet] plot ({\x},{ln(\x+2)}) node[above] {$y=\ln(x+2)$};
    \draw[domain=-2:0.95,variable=\x, orange] plot ({\x},{ln(1-\x)}) node[right] {$y=\ln(1-x)$};
    \draw[domain=0.05:3,variable=\x, black, dotted] plot ({\x},{ln(\x)}) node[right] {$y=\ln(x)$};
  \end{tikzpicture}
\end{center}
\end{cor}
\color{black}
\fi

%%%%%%%%%%%%%%%%


\begin{exo} Pour des r\'eels $a$, $b$ et $\beta$ ($\beta \neq 0$), calculer la d\'eriv\'ee des fonctions suivantes ($x$ est la variable):
\begin{multicols}{4}
\begin{enumerate}
\item $a^{x^2}$ ($a>0$)
\item $(ax+b)^{\beta}$
\item $\ln(ax+b)$
\item $e^{ax+b}$
\item $\cos(ax+b)$
\item $\sin(ax+b)$
\item $\tan(x)$
\item $\frac{e^x + e^{-x}}{2}$
\item $\frac{e^x - e^{-x}}{2}$
%\item $\frac{e^x - e^{-x}}{e^x + e^{-x}}$
\item $\arcsin(x)$
\item $\arccos(x)$
\item $\arctan(x)$
\end{enumerate}
\end{multicols}
\end{exo}

%%%%%%%%%%%%%%%%%
\ifcorrige
\color{magenta}
\begin{cor}
$\qquad$
\begin{enumerate}
\item Il s'agit de la compos\'ee des fonctions $f(x) = a^x = e^{x \ln(a)}$ et $g(x) = x^2$.
  Leurs d\'eriv\'ees valent $f'(x) = \ln(a) a^x$ et $g'(x) = 2x$.
  La d\'eriv\'ee de la fonction $a^{x^2} = (f \circ g)(x)$ vaut donc
  $f'(g(x)) \times g'(x) = 2 \ln(a) x a^{x^2}$.

\item On a $(ax+b)^{\beta} = e^{\beta \ln(ax+b) } = (f \circ g)(x)$
  avec $f(x) = x^{\beta} = e^{\beta \ln(x)}$ et $g(x) = ax+b$.
  Les d\'eriv\'ees de ces fonctions valent
  $f'(x) = \beta e^{\beta \ln(x)} \times \frac1x = \beta x^{\beta - 1}$
  et $g'(x) = a$.
  La d\'eriv\'ee de cette fonction vaut donc
  $f'(g(x)) \times g'(x) = \beta a (ax+b)^{\beta - 1}$.

\item On note $f(x) = \ln(ax+b)$.
  Par d\'eriv\'ee d'une compos\'ee de fonctions, on a $f'(x) = \frac{a}{ax+b}$.

\item On note $f(x) = e^{ax+b}$. Par deriv\'ee d'une compos\'ee de fonctions, on a
  $f'(x) = a e^{ax+b}$.

\item Idem:
  $- a \sin(ax+b)$

\item Idem: $a \cos(ax+b)$

\item On \'ecrit la fonction $\tan$ comme $\tan(x) = \frac{\sin(x)}{\cos(x)}$.
  En d\'erivant le quotient de fonctions, on a
  \begin{align*}
    \tan'(x) = \frac{\sin'(x)}{\cos(x)} - \frac{\cos'(x)\sin(x)}{\cos^2(x)}
    = 1 - \frac{\sin^2(x)}{\cos^2(x)} = 1 - \tan^2(x) .
  \end{align*}

\item Il s'agit de la fonction $\cosh(x) = \frac{e^x + e^{-x}}{2}$.
  On obtient facilement $\cosh'(x) = \frac{e^x - e^{-x}}{2} = \sinh(x)$.

\item Il s'agit de la fonction $\sinh(x) = \frac{e^x - e^{-x}}{2}$.
  On obtient facilement $\sinh'(x) = \frac{e^x + e^{-x}}{2} = \cosh(x)$.

\item L'arcsinus est d\'efini comme \'etant la fonction inverse du sinus
  (le sinus \'etant une bijection de $[-\frac{\pi}2,\frac{\pi}2]$ vers $[-1,1]$).
  Pour tout $x$ dans $[-\frac{\pi}2,\frac{\pi}2]$ on a donc $\arcsin(\sin(x)) = x$.
  En d\'erivant cette expression, on obtient
  $\arcsin'(\sin(x)) \cos(x) = 1$
  et donc
  $\arcsin'(\sin(x)) = \frac{1}{\sqrt{1-\sin^2(x)}}$.
  Notez que l'on a utilis\'e la formule de trigonom\'etrie $\cos^2(x) + \sin^2(x) = 1$
  et $\cos(x) \geq 0$ pour $x \in [-\frac{\pi}2,\frac{\pi}2]$.
  Par changement de variables, on a alors
  $\arcsin'(x) = \frac{1}{\sqrt{1-x^2}}$ pour tout $x \in ]-1,1[$.

  Notez que l'on peut aussi utiliser la formule de la d\'eriv\'ee de l'inverse d'une fonction:
  $(f^{-1})'(x) = \frac{1}{f'(f^{-1}(x))} = \frac{1}{\cos(\arcsin(x))}$ avec $f(x) = \sin(x)$.
  De plus, pour compl\'eter cette d\'emonstration, on peut montrer que $\cos(\arcsin(x)) = \sqrt{1 - x^2}$ :
  ceci vient de $\cos^2(t) + \sin^2(t) = 1$ combin\'e avec le changement de variables
  $t = \arcsin(x)$.

\item L'arccosinus est d\'efini comme \'etant l'inverse de la fonction cosinus
  (le cosinus \'etant une bijection de $[0,\pi]$ vers $[-1,1]$).
  On peut utiliser la m\^eme m\'ethodologie qu'\`a la question pr\'ec\'edente.
  Pour tout $x$ dans $[0,\pi]$, on a $\arccos(\cos(x)) = x$. En d\'erivant:
  $- \sin(x) \arccos'(\cos(x)) = 1$ et donc
  $\arccos'(\cos(x)) = \frac{-1}{\sin(x)} = \frac{-1}{\sqrt{1-\cos^2(x)}}$
  et donc $\arccos'(x) = \frac{-1}{\sqrt{1-x^2}}$.

  Comme pour la question pr\'ec\'edente, on peut aussi utiliser la formule de la d\'eriv\'ee
  de la fonction inverse.

\item De la m\^eme fa\c{c}on :
  $\arctan(\tan(x)) = x$
  et donc
  $\arctan'(\tan(x)) (1+\tan^2(x)) = 1$
  et donc $\arctan'(x) = \frac{1}{1+x^2}$.
\end{enumerate}
\end{cor}
\color{black}
\fi

%%%%%%%%%%%%%%%%


\begin{exo} Calculer la limite en $+\infty$ des fonctions suivantes:
\begin{multicols}{3}
\begin{enumerate}
\item $3x^4 - x^3 + 5x^2 + x - 1$
\item $\frac{3x^2-2x+1}{x+4}$
\item $\frac{x^3-4x^2+1}{x^5+2}$
\item $\frac{3x+\sqrt{x}}{x-1}$
\item $\sqrt{x^2+4x-1} - 2x$
\item $\sqrt{x+1} - \sqrt{x-1}$
\end{enumerate}
\end{multicols}
\end{exo}

%%%%%%%%%%%%%%%%%
\ifcorrige
\color{magenta}
\begin{cor}
$\qquad$
\begin{enumerate}
\item La limite en l'infini d'un polyn\^ome est donn\'ee par son mon\^ome de plus haut degr\'e.
  Comme $\lim_{x \to + \infty} 3x^4 = + \infty$,
  alors on a $\lim_{x \to + \infty} 3x^4  - x^3 + 5x^2 + x - 1= + \infty$.

\item La limite en l'infini d'un quotient de polyn\^omes est donn\'ee par la limite
  du quotient des mon\^omes de plus haut degr\'e.
  Ainsi, $\lim_{x \to +\infty}\frac{3x^2-2x+1}{x+4} = \lim_{x \to +\infty}\frac{3x^2}{x} = + \infty$

\item Idem:
  $\lim_{x \to + \infty} \frac{x^3-4x^2+1}{x^5+2} = \lim_{x \to + \infty} \frac{x^3}{x^5} = 0$.

\item Ici on peut s\'eparer la fraction en deux :
  $\frac{3x+\sqrt{x}}{x-1} = \frac{3x}{x-1} + \frac{\sqrt{x}}{x-1}$.
  De plus, $\lim_{x \to + \infty}\frac{3x}{x-1} = 3$
  et $\lim_{x \to + \infty}\frac{\sqrt{x}}{x-1}
  = \lim_{x \to + \infty}\frac{1}{\sqrt{x}-\frac1{\sqrt{x}}} = 0$.
  Ainsi, en sommant les deux limites : $\lim_{x \to +\infty}\frac{3x+\sqrt{x}}{x-1} = 3$.

\item On peut traiter ce cas en majorant la fonction.
  On a, pour $x$ suffisamment grand, $x^2 + 4x - 1 \leq \frac94 x^2$ et donc
  (par croissance de la racine carr\'ee)
  $\sqrt{x^2 + 4x - 1} \leq \frac32 x$.
  Ainsi, pour $x$ suffisamment grand, $\sqrt{x^2 + 4x - 1} - 2x \leq - \frac{x}2$.
  On a, de plus, $\lim_{x \to +\infty} - \frac{x}2 = -\infty$.
  Par majoration, on a donc $\lim_{x \to +\infty} \sqrt{x^2 + 4x - 1} - 2x = -\infty$.

\item Notons $f(x) = \sqrt{x}$.
  On cherche donc la limite en $+\infty$ de $f(x+1) - f(x-1)$.
  D'apr\`es l'in\'egalit\'e des accroissements finis,
  on a pour $x \geq a$, $| f(x+1) - f(x-1) | \leq 2 \sup_{z \in [x-1, x+1]} |f'(z)|$.
  Or $f'(z) = \frac{1}{2\sqrt{z}}$ et ainsi $\lim_{x \to +\infty} \sup_{z \in [x-1, x+1]} |f'(z)| = 0$
  et donc $\lim_{x \to +\infty} f(x+1) - f(x-1) = 0$.
  
\end{enumerate}
\end{cor}
\color{black}
\fi

%%%%%%%%%%%%%%%%


\begin{exo} Calculer les limites suivantes:
\begin{multicols}{2}
\begin{enumerate}
\item de $\frac{x^2-4x+3}{x^2+3x-4}$ en $x=1$
\item de $\frac{\cos(x) - 1}{x}$ en $x=0$
\end{enumerate}
\end{multicols}
\end{exo}

%%%%%%%%%%%%%%%%%
\ifcorrige
\color{magenta}
\begin{cor}
$\qquad$
\begin{enumerate}
\item Si la fonction \'etait d\'efinie en $1$, il suffirait de l'\'evaluer pour obteir sa limite.
  Ici cette fonction est une fraction, le num\'erateur et le d\'enominateur tendent tous les deux vers z\'ero
  quand $x \to 1$ (et ce sont des polyn\^omes).
  On peut donc mettre $x-1$ en facteur et simplifier.
  On a $x^2-4x+3 = (x-1)(x-3)$ et $x^2+3x-4 = (x-1)(x+4)$.
  Ainsi, $\frac{x^2-4x+3}{x^2+3x-4} = \frac{x-3}{x+4}$
  et donc $\lim_{x \to 1} \frac{x^2-4x+3}{x^2+3x-4} = \lim_{x \to 1} \frac{x-3}{x+4} = - \frac25$.

\item Pour cette question, nous allons faire un d\'eveloppement limit\'e du cosinus en $0$:
  $\cos(x) = 1 - \frac{x^2}{2} + O(x^4)$ et donc
  $\frac{\cos(x) - 1}{x} = - \frac{x}{2} + O(x^3)$
  et donc $\lim_{x \to 0} \frac{\cos(x)-1}{x} = 0$.
\end{enumerate}
\end{cor}
\color{black}
\fi

%%%%%%%%%%%%%%%%


\begin{exo} \'Etudier les fonctions suivantes et tracer leur graphe:
\begin{multicols}{4}
\begin{enumerate}
\item $\frac{x}{x-2}$
\item $\cosh(x)$
\item $\sinh(x)$
\item $\tanh(x)$
\end{enumerate}
\end{multicols}
\end{exo}

%%%%%%%%%%%%%%%%%
\ifcorrige
\color{magenta}
\begin{cor}
$\qquad$
\begin{enumerate}
\item Cette fonction $f$ est d\'efinie sur $]-\infty,2[\cup]2,+\infty[$.
  $\lim_{x \to 2^-} \frac{x}{x-2} = - \infty$
  et $\lim_{x \to 2^+} \frac{x}{x-2} = + \infty$.
  Sa d\'eriv\'ee vaut
  $f'(x) = \frac{x-2 - x}{(x-2)^2} = \frac{-2}{(x-2)^2} < 0$.

\item La fonction $\cosh(x) = \frac{e^x + e^{-x}}{2}$ est d\'efinie sur $\R$ entier.
  Sa d\'eriv\'ee vaut $\sinh(x)$ qui est n\'egative sur $\R_-^*$ et positive sur $\R_+^*$.
  De plus, on remarque que la fonction $\cosh$ est paire.

\item La fonction $\sinh(x) = \frac{e^x - e^{-x}}{2}$ est d\'efinie sur $\R$ entier.
  Sa d\'eriv\'ee vaut $\cosh(x)$ qui est toujours positive.
  De plus, on remarque que la fonction $\sinh$ est impaire.

\item La fonction $\tanh(x) = \frac{\sinh(x)}{\cosh(x)} = \frac{e^x - e^{-x}}{e^x + e^{-x}}$
  est d\'efinie sur $\R$ entier.
  Sa d\'eriv\'ee vaut $\tanh'(x) = 1 - \tanh^2(x)$.
  De plus, on remarque que $-1 < \tanh(x) < 1$ pour tout $x \in \R$
  et donc cette fonction est strictement croissante.
  On peut de plus montrer qu'elle est impaire.

\item Trac\'e des fonctions :
    \begin{center}
  \begin{tikzpicture}[scale=2]
    \draw[->] (-1,0) -- (3,0) node[right] {$x$};
    \draw[->] (0,-2) -- (0,3) node[above] {$y$};
    \draw[domain=-3:1.5,variable=\x, green] plot ({\x},{\x / (\x-2)}) node[above] {$y=\frac{x}{x-2}$};
    \draw[domain=2.5:3,variable=\x, green] plot ({\x},{\x / (\x-2)}) node[above] {$y=\frac{x}{x-2}$};
    \draw[domain=-2:2,variable=\x, blue] plot ({\x},{cosh(\x)}) node[above] {$y=\cosh(x)$};
    \draw[domain=-2:2,variable=\x, red] plot ({\x},{sinh(\x)}) node[left] {$y=\sinh(x)$};
    \draw[domain=-3:3,variable=\x, violet] plot ({\x},{tanh(\x)}) node[above] {$y=\tanh(x)$};
  \end{tikzpicture}
\end{center}
\end{enumerate}
\end{cor}
\color{black}
\fi

%%%%%%%%%%%%%%%%


\begin{exo} Soit une constante $\alpha \in \R^*$. Donner une primitive de chacune des fonctions suivantes:
\begin{multicols}{4}
\begin{enumerate}
\item $\sin(\alpha x)$
\item $\cos(\alpha x)$
\item $\frac{1}{\alpha x}$
\item $e^{-\alpha x}$
\item $x^{\frac1{\alpha}}$
\item $\sqrt{\alpha x}$
\item $\frac1{\sqrt{\alpha x}}$
\item $\frac{1}{\sqrt{\alpha^2-x^2}}$
\item $\frac{1}{\alpha^2 + x^2}$
\item $\ln(x)$
\item $x \ln(x)$
\item $x \cos(x)$
\end{enumerate}
\end{multicols}
\end{exo}

%%%%%%%%%%%%%%%%%
\ifcorrige
\color{magenta}
\begin{cor}
Tous les r\'esultats de cette section peuvent \^etre v\'erifi\'es en d\'erivant la primitive
pour retrouver la fonction initiale.
\begin{enumerate}
\item $- \frac{1}{\alpha} \cos(\alpha x)$
\item $\frac{1}{\alpha} \sin(\alpha x)$
\item $\frac{1}{|\alpha|} \ln(|\alpha x|)$
  Attention pour celui-ci : la fonction $\frac{1}{\alpha x}$ est d\'efinie sur $\R^*$
  tandis que le logarithme est d\'efini seulement sur $\R_+^*$.
  Il faut donc pouvoir envisager le cas $\alpha x<0$ et c'est pour cela qu'il ne faut pas oublier les valeurs
  absolues.
\item $- \frac{1}{\alpha}e^{-\alpha x}$
\item Si $\alpha \neq -1$ alors la primitive vaut $\frac{\alpha}{\alpha+1} x^{1+\frac{1}{\alpha}}$.
  Si $\alpha = -1$, alors la primitive de $\frac{1}{x}$ est $\ln(|x|)$.
\item $\sqrt{\alpha x} = (\alpha x)^{\frac12}$ et donc la primitive de cette fonction vaut
  $\frac2{3\alpha} (\alpha x)^{\frac32}$.
\item De m\^eme, $\frac{1}{\sqrt{\alpha x}} = (\alpha x)^{-\frac12}$
  et donc sa primitive vaut
  $\frac{2}{\alpha}(\alpha x)^{\frac12}$.
\item Pour cette primitive, il faut reconna\^itre la d\'eriv\'ee de l'arcsinus :
  $\arcsin'(x) = \frac{1}{\sqrt{1-x^2}}$.
  Ainsi, on a $\frac{1}{\sqrt{\alpha^2-x^2}} = \frac{1}{|\alpha|\sqrt{1-(\frac{x}{\alpha})^2}}$.
  Et donc la primitive recherch\'ee vaut
  $\frac{\alpha}{|\alpha|}\arcsin(\frac{x}{\alpha})$.
\item On rappelle la d\'eriv\'ee de l'arctangente: $\arctan'(x) = \frac{1}{1+x^2}$.
  De plus, $\frac{1}{\alpha^2 + x^2} = \frac{1}{\alpha^2(1 + (\frac{x}{\alpha})^2)}$
  et donc la primitive de cette fonction vaut
  $\frac{1}{\alpha}\arctan(\frac{x}{\alpha})$.
\item On utilise une int\'egration par parties pour cette primitive (on d\'erive $\ln(t)$ et on primitive $1$)
  \begin{align*}
    \int^x 1 \ln(t) = [t \ln(t)]^x - \int^x 1 = x \ln(x) - x
  \end{align*}
  Une primitive de $\ln(x)$ est donc $x \ln(x) - x$
  (on peut v\'erifier ce r\'esultat en d\'erivant cette fonction).

\item A nouveau, on utilise une int\'egration par parties
  (on d\'erive $\ln(t)$ et on primitive $t$)
  \begin{align*}
    \int^x t \ln(t) = [\frac{t^2}{2} \ln(t)]^x - \int^x \frac{t^2}{2} \times \frac{1}{t}
    = \frac{x^2}{2} \ln(x) - \frac{x^2}{4}
  \end{align*}

\item Encore une fois, on utilise une int\'egration par parties.
  On d\'erive $t$ et on primitive $\cos(t)$:
  \begin{align*}
    \int^x t \cos(t) = [t\sin(t)]^x - \int^x \sin(t)
    = x \sin(x) + \cos(x)
  \end{align*}
\end{enumerate}
\end{cor}
\color{black}
\fi

%%%%%%%%%%%%%%%%



\begin{exo} Calculer les int\'egrales suivantes:
\begin{multicols}{3}
\begin{enumerate}
\item $\int_0^{\pi} \sin(x) \;dx$
\item $\int_0^{+\infty} e^{-x} \;dx$
\item $\int_0^{\pi} x \sin(x) \;dx$
\item $\int_0^1 x e^x \;dx$
\item $\int_{- \sqrt{\frac{\pi}{2}}}^{\sqrt{2\pi}} 2x \cos(x^2) \;dx$
\item $\int_{0}^1 \frac{e^x - 1}{e^x + 1} \;dx$
\end{enumerate}
\end{multicols}
\end{exo}

%%%%%%%%%%%%%%%%%
\ifcorrige
\color{magenta}
\begin{cor}
$\qquad$
\begin{enumerate}
\item Ici, on conna\^it une primitive de l'int\'egrande.
  $\int_0^{\pi} \sin(x) \;dx = [-\cos(x)]_0^{\pi} = -\cos(\pi) + \cos(0) = 2$
\item Idem. $\int_0^{+\infty} e^{-x} \;dx = [-e^{-x}]_0^{+\infty} = 1$
\item Ici, on utilise une int\'egration par parties.
  $\int_0^{\pi} x \sin(x) \;dx = [-x \cos(x)]_0^{\pi} - \int_0^{\pi} - \cos(x) \;dx
  = \pi + [\sin(x)]_0^{\pi} = \pi$
\item Idem.
  $\int_0^1 x e^x \;dx = [xe^x]_0^1 - \int_0^1 e^x \;dx
  = e - e + 1 = 1$

\item Ici, on utilise le changement de variables
  $t = x^2$ et donc $dt = 2x dx$
  \begin{align*}
    \int_{- \sqrt{\frac{\pi}{2}}}^{\sqrt{2\pi}} 2x \cos(x^2) \;dx
    &= \int_{- \sqrt{\frac{\pi}{2}}}^0 2x \cos(x^2) \;dx
    + \int_0^{\sqrt{2\pi}} 2x \cos(x^2) \;dx
    \\
    &= \int_{\frac{\pi}{2}}^0 \cos(t) \;dt
    + \int_0^{2\pi} \cos(t) \;dt
    = \int_{\frac{\pi}{2}}^{2\pi} \cos(t) \;dt
    = [\sin(t)]_{\frac{\pi}{2}}^{2\pi}
    = 0 - 1
    \\
    &= -1
  \end{align*}

\item
  Ici, on utilise le changement de variables
  $t = e^x$ et donc $dt = e^x \;dx$ ce qui donne $dx = \frac{dt}{t}$.
  \begin{align*}
    \int_{0}^1 \frac{e^x - 1}{e^x + 1} \;dx
    = \int_{1}^e \frac{t - 1}{t(t + 1)} \;dt
  \end{align*}
  Maintenant, il faut trouver $a$ et $b$ tels que
  \begin{align*}
    \frac{t - 1}{t(t + 1)}
    =
    \frac{a}{t} + \frac{b}{t+1}
    =
    \frac{(a+b)t+a}{t(t+1)}
  \end{align*}
  et donc $a=-1$ et $b=2$.
  \begin{align*}
    \int_{0}^1 \frac{e^x - 1}{e^x + 1} \;dx
    = \int_{1}^e \frac{t - 1}{t(t + 1)} \;dt
    = \int_{1}^e \frac{- 1}{t} + \frac{2}{t+1} \;dt
    = [-\ln(t) + 2 \ln(t+1)]_1^e
    \\
    = -1 + 2 \ln(e+1) - 2 \ln(2)
  \end{align*}
\end{enumerate}
\end{cor}
\color{black}
\fi

%%%%%%%%%%%%%%%%
\end{document}
